% Reproducible Builds
\pdftrailerid{}

\usepackage{tabu}
\usepackage{tcolorbox}

% Color stuff
\usepackage{xcolor}
\usepackage{colortbl}
% Gruvbox
\definecolor{GbBg}{HTML}{fbf1c7}
\definecolor{GbBg0}{HTML}{f9f5d7}
\definecolor{GbBg00}{HTML}{fbf8e4}
\definecolor{GbFg0}{HTML}{282828}
\definecolor{GbFg1}{HTML}{3c3836}
\definecolor{GbFg2}{HTML}{504945}
\definecolor{GbFg3}{HTML}{665c54}
\definecolor{GbFg4}{HTML}{7c6f64}
\definecolor{GbAquaDk}{HTML}{427b58}
\definecolor{GbAquaNt}{HTML}{689da6}
\definecolor{GbBlueDk}{HTML}{076678}
\definecolor{GbBlueNt}{HTML}{458588}
\definecolor{GbGreenDk}{HTML}{79740e}
\definecolor{GbGreenNt}{HTML}{98971a}
\definecolor{GbGreyDk}{HTML}{3c3836}
\definecolor{GbGrey}{HTML}{504945}
\definecolor{GbGreyNt}{HTML}{7c6f64}
\definecolor{GbOrangeNt}{HTML}{d65d0e}
\definecolor{GbPurpleDk}{HTML}{8f3f71}
\definecolor{GbPurpleNt}{HTML}{b16286}
\definecolor{GbRedDk}{HTML}{9d0006}
\definecolor{GbRedNt}{HTML}{cc241d}
\definecolor{GbYellowDk}{HTML}{b57614}
\definecolor{GbyellowNt}{HTML}{d79921}
\usepackage{mathtools}

\usepackage{subcaption}
\usepackage{pgfmath}
\usepackage{tikz}
\usetikzlibrary{shapes.geometric, arrows, positioning, calc, patterns}

\tikzstyle{process} = [rectangle, minimum width=2cm, minimum height=1cm, text centered, draw=black]
\tikzstyle{data} = [trapezium, trapezium left angle=70, trapezium right angle=110, minimum width=1.5cm, minimum height=1cm, text centered, draw=black]
\tikzstyle{arrow} = [->, >=latex]
\tikzstyle{thick-arrow} = [thick, ->, >=latex]
\tikzstyle{dashed-arrow} = [dashed, ->]
\tikzstyle{node} = [rectangle, draw=black, minimum width=1cm, minimum height=1cm, text centered]
\tikzstyle{box} = [rectangle, draw, rounded corners, text centered, minimum height=1cm]

\usepackage{pgfplots}
\pgfplotsset{compat=1.18}
\usepackage{pgfplotstable}

\usepackage{hyperref}
\hypersetup{
    colorlinks=true, %set true if you want colored links
    linktoc=all,     %set to all if you want both sections and subsections linked
    linkcolor=GbBlueNt,    % Internal links (e.g., section links)
    citecolor=GbGreenNt,   % Citation links
    urlcolor=GbBlueDk      % External URL links
}
\usepackage{array}
\usepackage{caption}
\usepackage{algorithm}
\usepackage{algpseudocode}
\usepackage{stmaryrd}
\usepackage{multicol}
\usepackage{multirow}
%\usepackage{MnSymbol}
\usepackage{yfonts}
\usepackage{amsmath, amssymb}
\usepackage{graphicx}
\usepackage{xparse}
\usepackage{pgffor}
\usepackage{booktabs}
\usepackage{etoolbox}
\usepackage{longtable}
\usepackage{syntax}

\usepackage{tabularx}
\newcolumntype{Y}{>{\centering\arraybackslash}X}

\newcommand{\Desc}[2]{\hspace*{\algorithmicindent} \makebox[14.5em][l]{#1}\parbox[t]{32em}{#2}}

\usepackage[style=iso]{datetime2}

% Float figures
\usepackage{float}
\let\origfigure\figure
\let\endorigfigure\endfigure
\renewenvironment{figure}[1][2] {
    \expandafter\origfigure\expandafter[H]
} {
    \endorigfigure
}

% Circuit Diagram Macros
\tikzset{
  port/.pic={
    % Default height = 1cm (can be scaled using pic scale)
    \def\h{0.125}
    \pgfmathsetmacro{\b}{2 / sqrt(3) * \h} % base length from height
    \pgfmathsetmacro{\bx}{\b / 2} % half-base x offset

    % Vertices: tip at (0,0), base at height h
    \draw[pic actions, fill=black]
      (0,0) -- (-\bx,\h) -- (\bx,\h) -- cycle;
  }
}
% TikZ Gate Macro - Clean Version
% Usage: \gate{id}{coordinate}{input_labels}{gate_operation}{num_outputs}
% Example: \gate{id}{(0,0)}{$\abst{x}_1$,$\abst{x}_2$,$\abst{x}_3$}{$\ty(g)$}{2}
% For no inputs: \gate{input1}{(0,0)}{}{$\text{Input}^q_1$}{1}
\makeatletter

% Helper to check if string is empty
\newcommand{\gate@ifempty}[3]{%
  \if\relax\detokenize{#1}\relax
    #2%
  \else
    #3%
  \fi
}

% Main gate command
\newcommand{\gate}[5]{%
  \begin{scope}[shift={#2}, local bounding box=#1]
    \begin{scope}[local bounding box=#1-top]
      \gate@ifempty{#3}{%
        % No input labels - just place gate operation at origin
        \node[anchor=north west] (#1-g) at (0,0) {#4};
      }{%
        % Process input labels
        \begin{scope}[local bounding box=#1-in]
          \expandafter\gate@process@inputs\expandafter{#1}{#3}
        \end{scope}
        % Draw internal borders
        \expandafter\gate@draw@borders\expandafter{#1}{#3}
        % Bottom border of input section
        \draw (#1-in.south west) -- (#1-in.south east);
        % Gate operation below inputs
        \node[anchor=north] (#1-g) at (#1-in.south) {#4};
      }
    \end{scope}
    % Output ports
    \gate@draw@outputs{#1}{#5}
  \end{scope}
  \draw[rounded corners] (#1.north west) rectangle (#1.south east);
}

% Process comma-separated input labels
\newcommand{\gate@process@inputs}[2]{%
  \def\gate@current@input{1}%
  \def\gate@port@height{0.125}% Height of port pic
  \gate@process@input@list{#1}#2,\gate@end@marker,%
}

\def\gate@process@input@list#1#2,#3\gate@end@marker{%
  \ifnum\gate@current@input=1
    % First input: place label below the port height to align top-left corners
    \node[anchor=north west] (#1-in\gate@current@input l) at (0,-\gate@port@height) {#2};
  \else
    \pgfmathtruncatemacro{\gate@prev@input}{\gate@current@input-1}
    \node[anchor=north west] (#1-in\gate@current@input l) at (#1-in\gate@prev@input l.north east) {#2};
  \fi
  % Coordinate at the base of the input port (tip + port height)
  \coordinate (#1-in-\gate@current@input) at ($(#1-in\gate@current@input l.north)+(0,\gate@port@height)$);
  \pic[anchor=south] at (#1-in\gate@current@input l.north) {port};
  \pgfmathtruncatemacro{\gate@current@input}{\gate@current@input+1}
  \ifx\relax#3\relax\else
    \gate@process@input@list{#1}#3\gate@end@marker
  \fi
}

% Count inputs first
\newcommand{\gate@count@inputs}[1]{%
  \def\gate@input@count{0}%
  \gate@count@input@list#1,\gate@end@marker,%
}

\def\gate@count@input@list#1,#2\gate@end@marker{%
  \pgfmathtruncatemacro{\gate@input@count}{\gate@input@count+1}
  \ifx\relax#2\relax\else
    \gate@count@input@list#2\gate@end@marker
  \fi
}

% Draw vertical borders between inputs
\newcommand{\gate@draw@borders}[2]{%
  \gate@count@inputs{#2}%
  \ifnum\gate@input@count>1
    \foreach \i in {2,...,\gate@input@count} {
      \draw (#1-in\i l.north west |- #1-in.north west) -- (#1-in\i l.south west |- #1-in.south west);
    }
  \fi
}

% Draw output ports
\newcommand{\gate@draw@outputs}[2]{%
  \ifnum#2>0
    \foreach \i in {1,...,#2} {
      \pgfmathsetmacro{\gate@output@pos}{(\i-0.5)/#2}
      % Coordinate at the base of the output port (tip - port height since yscale=-1)
      \coordinate (#1-out-\i) at ($(#1-top.south west)!\gate@output@pos!(#1-top.south east)+(0,-0.125)$);
      \pic[yscale=-1, anchor=north] at ($(#1-top.south west)!\gate@output@pos!(#1-top.south east)$) {port};
    }
  \fi
}

\makeatother

% \newcommand*\Bb{\mathbb{B}}
% \newcommand*\Zb{\mathbb{Z}}
% \newcommand*\Fb{\mathbb{F}}
% \newcommand*\Nb{\mathbb{N}}
% \newcommand*\Rb{\mathbb{R}}
% \newcommand*\Eb{\mathbb{E}}
% \newcommand*\Gb{\mathbb{G}}
% \newcommand*\Ac{\mathcal{A}}
% \newcommand*\Ec{\mathcal{E}}
% \newcommand*\Rc{\mathcal{R}}
% \newcommand*\Ic{\mathcal{I}}
% \newcommand*\Oc{\mathcal{O}}
% \newcommand*\Uc{\mathcal{U}}
% \newcommand*\Mc{\mathcal{M}}
% \newcommand*\Pc{\mathcal{P}}
% \newcommand*\Vc{\mathcal{V}}
% \newcommand*\Sc{\mathcal{S}}
% \newcommand*\Hc{\mathcal{H}}

% \renewcommand*\a{\alpha}
% \renewcommand*\b{\beta}
% \renewcommand*\d{\delta}
% \newcommand*\e{\epsilon}
% \newcommand*\g{\gamma}
% \renewcommand*\l{\lambda}
% \newcommand*\p{\phi}
% \renewcommand*\o{\omega}
% \newcommand*\s{\sigma}
% \newcommand*\ps{\psi}
% \renewcommand*\S{\Sigma}

% \renewcommand*\mod{\bmod}
% \newcommand*\cat{\mathbin{+\mkern-10mu+}}
% \newcommand*\bor{\mathbin{\&\mkern-7mu\&}}
% \newcommand*\xor{\oplus}
% \newcommand*\meq{\stackrel{?}{=}}
% \newcommand*\iso{\cong}
% \newcommand{\qed}{\hfill \ensuremath{\Box}}
% \newcommand{\defend}{\hfill \ensuremath{\triangle}}
% \newcommand*{\then}{\implies}
% \newcommand*{\from}{\leftarrow}

% \newcommand*\algind{\hspace*{\algorithmicindent}}
% \newcommand*\algindd{\algind \algind}

% \newcommand{\textblue}[1]{\textcolor{GbBlueDk}{#1}}
% \newcommand{\mathblue}[1]{\mathcolor{GbBlueDk}{#1}}
% \newcommand{\floor}[1]{\left \lfloor #1 \right \rfloor }
% \newcommand{\ceil}[1]{\left \lceil #1 \right \rceil }
% \renewcommand{\vec}[1]{ \boldsymbol{#1} }
% \newcommand{\ran}[1]{ \mathrm{#1} }
% \newcommand{\ranvec}[1]{ \boldsymbol{\ran{#1}} }
% \newcommand{\dotp}[2]{ \langle #1, #2 \rangle }
% \newcommand{\ip}[2]{ \langle #1, #2 \rangle }

% \newcommand{\negl}{\text{negl}}
% \newcommand{\poly}{\text{poly}}

% \newcommand*{\pp}{\text{pp}}
% \newcommand*{\acc}{\text{acc}}
% \newcommand*{\Prover}{\mathrm{\text{P\scriptsize ROVER}}}
% \newcommand*{\Verifier}{\mathrm{\text{V\scriptsize ERIFIER}}}
% \newcommand*{\Setup}{\mathrm{\text{S{\scriptsize ETUP}}}}
% \newcommand*{\Generator}{\mathrm{\text{G{\scriptsize ENERATOR}}}}
% \newcommand*{\Trim}{\mathrm{\text{T{\scriptsize RIM}}}}
% \newcommand*{\Commit}{\mathrm{\text{C{\scriptsize OMMIT}}}}
% \newcommand*{\Decider}{\mathrm{\text{D{\scriptsize ECIDER}}}}
% \newcommand*{\SNARKProver}{\mathrm{\text{SNARK}}.\Prover}
% \newcommand*{\SNARKVerifier}{\mathrm{\text{SNARK}}.\Verifier}
% \newcommand*{\SNARKVerifierSlow}{\mathrm{\text{SNARK}}.\mathrm{\text{V\scriptsize ERIFIER}\text{S\scriptsize LOW}}}
% \newcommand*{\IVCProver}{\mathrm{\text{IVC}}.\Prover}
% \newcommand*{\IVCVerifier}{\mathrm{\text{IVC}}.\Verifier}
% \newcommand*{\AS}{\text{AS}}
% \newcommand*{\ASGenerator}{\AS.\Generator}
% \newcommand*{\ASProver}{\AS.\Prover}
% \newcommand*{\ASVerifier}{\AS.\Verifier}
% \newcommand*{\ASDecider}{\AS.\Decider}
% \newcommand*{\PC}{\text{PC}}
% \newcommand*{\PCSetup}{\PC.\Setup}
% \newcommand*{\PCTrim}{\PC.\Trim}
% \newcommand*{\PCCommit}{\PC.\Commit}
% \newcommand*{\PCOpen}{\PC.\mathrm{\text{O\scriptsize PEN}}}
% \newcommand*{\PCCheck}{\PC.\mathrm{\text{C\scriptsize HECK}}}
% \newcommand*{\PCDL}{\text{PC}_{\text{DL}}}
% \newcommand*{\PCDLSetup}{\PCDL.\Setup}
% \newcommand*{\PCDLTrim}{\PCDL.\Trim}
% \newcommand*{\PCDLCommit}{\PCDL.\Commit}
% \newcommand*{\PCDLOpen}{\PCDL.\mathrm{\text{O\scriptsize PEN}}}
% \newcommand*{\PCDLSuccinctCheck}{\PCDL.\mathrm{\text{S{\scriptsize UCCINCT}C{\scriptsize HECK}}}}
% \newcommand*{\PCDLCheck}{\PCDL.\mathrm{\text{C\scriptsize HECK}}}
% \newcommand*{\ASDL}{\text{AS}_{\text{DL}}}
% \newcommand*{\ASDLGenerator}{\ASDL.\Generator}
% \newcommand*{\ASDLIndexer}{\ASDL.\mathrm{\text{I\scriptsize NDEXER}}}
% \newcommand*{\ASDLCommonSubroutine}{\ASDL.\mathrm{\text{C{\scriptsize OMMON}S{\scriptsize UBROUTINE}}}}
% \newcommand*{\ASDLProver}{\ASDL.\mathrm{\text{P\scriptsize ROVER}}}
% \newcommand*{\ASDLVerifier}{\ASDL.\mathrm{\text{V\scriptsize ERIFIER}}}
% \newcommand*{\ASDLDecider}{\ASDL.\mathrm{\text{D\scriptsize ECIDER}}}
% \newcommand*{\CM}{\mathrm{\text{CM}}}
% \newcommand*{\CMSetup}{\CM.\Setup}
% \newcommand*{\CMTrim}{\CM.\Trim}
% \newcommand*{\CMCommit}{\CM.\Commit}

% \newcommand*\Result{\mathbf{Result}}
% \newcommand*\Option{\mathbf{Option}}
% \newcommand*\Acc{\mathbf{Acc}}
% \newcommand*\AccHiding{\mathbf{AccHiding}}
% \newcommand*\EvalProof{\mathbf{EvalProof}}

\newcommand*\incompleteadd{\boxplus}

\newcommand*\Bb{\mathbb{B}}
\newcommand*\Zb{\mathbb{Z}}
\newcommand*\Fb{\mathbb{F}}
\newcommand*\Nb{\mathbb{N}}
\newcommand*\Rb{\mathbb{R}}
\newcommand*\Eb{\mathbb{E}}
\newcommand*\Gb{\mathbb{G}}
\newcommand*\Ac{\mathcal{A}}
\newcommand*\Bc{\mathcal{B}}
\newcommand*\Cc{\mathcal{C}}
\newcommand*\Dc{\mathcal{D}}
\newcommand*\Ec{\mathcal{E}}
\newcommand*\Rc{\mathcal{R}}
\newcommand*\Oc{\mathcal{O}}
\newcommand*\Uc{\mathcal{U}}
\newcommand*\Mc{\mathcal{M}}
\newcommand*\Pc{\mathcal{P}}
\newcommand*\Vc{\mathcal{V}}
\newcommand*\Sc{\mathcal{S}}
\newcommand*\Hc{\mathcal{H}}

\renewcommand*\a{\alpha}
\renewcommand*\b{\beta}
\renewcommand*\d{\delta}
\newcommand*\e{\epsilon}
\newcommand*\g{\gamma}
\renewcommand*\l{\lambda}
\newcommand*\p{\phi}
\newcommand*\s{\sigma}
\renewcommand*\o{\omega}
\newcommand*\ps{\psi}
\renewcommand*\S{\Sigma}

\renewcommand*\mod{\bmod}
\newcommand*\cat{\mathbin{+\mkern-10mu+}}
\newcommand*\bor{\mathbin{\&\mkern-7mu\&}}
\newcommand*\xor{\oplus}
\newcommand*\meq{\stackrel{?}{=}}
\newcommand*\iso{\cong}
\newcommand{\qed}{\hfill \ensuremath{\Box}}
\newcommand{\defend}{\hfill \ensuremath{\triangle}}
\newcommand*{\then}{\implies}

\newcommand*\algind{\hspace*{\algorithmicindent}}
\newcommand*\algindd{\algind \algind}

\newcommand{\textblue}[1]{\textcolor{GbBlueDk}{#1}}
\newcommand{\mathblue}[1]{\mathcolor{GbBlueDk}{#1}}
\newcommand{\textgrey}[1]{\textcolor{GbGrey}{#1}}
\newcommand{\mathgrey}[1]{\mathcolor{GbGrey}{#1}}
\newcommand{\floor}[1]{\left \lfloor #1 \right \rfloor }
\newcommand{\ceil}[1]{\left \lceil #1 \right \rceil }
\renewcommand{\vec}[1]{ \boldsymbol{#1} }
\newcommand{\ran}[1]{ \mathrm{#1} }
\newcommand{\ranvec}[1]{ \boldsymbol{\ran{#1}} }
\newcommand{\dotp}[2]{ \langle #1, #2 \rangle }
\newcommand{\ip}[2]{ \langle #1, #2 \rangle }
\newcommand{\IK}[2]{ \mathrm{I.K} }

\newcommand*{\negl}{\text{negl}}
\newcommand*{\poly}{\text{poly}}
\newcommand*{\from}{\gets}
\newcommand*{\pp}{\mathrm{pp}}
\newcommand*{\acc}{\mathrm{acc}}
\newcommand*{\ToInstance}{\mathrm{\text{T\scriptsize O}\text{I\scriptsize NSTANCE}}}
\newcommand*{\Prover}{\mathrm{\text{P\scriptsize ROVER}}}
\newcommand*{\Verifier}{\mathrm{\text{V\scriptsize ERIFIER}}}
\newcommand*{\Setup}{\mathrm{\text{S{\scriptsize ETUP}}}}
\newcommand*{\Generator}{\mathrm{\text{G{\scriptsize ENERATOR}}}}
\newcommand*{\Trim}{\mathrm{\text{T{\scriptsize RIM}}}}
\newcommand*{\Commit}{\mathrm{\text{C{\scriptsize OMMIT}}}}
\newcommand*{\Decider}{\mathrm{\text{D{\scriptsize ECIDER}}}}
\newcommand*{\SNARKProver}{\mathrm{\text{SNARK}}.\Prover}
\newcommand*{\SNARKVerifier}{\mathrm{\text{SNARK}}.\Verifier}
\newcommand*{\SNARKVerifierSlow}{\mathrm{\text{SNARK}}.\mathrm{\text{V\scriptsize ERIFIER}\text{S\scriptsize LOW}}}
\newcommand*{\SNARKVerifierFast}{\mathrm{\text{SNARK}}.\mathrm{\text{V\scriptsize ERIFIER}\text{F\scriptsize AST}}}
\newcommand*{\NARKProver}{\mathrm{\text{NARK}}.\Prover}
\newcommand*{\NARKVerifier}{\mathrm{\text{NARK}}.\Verifier}
\newcommand*{\NARKVerifierSlow}{\mathrm{\text{NARK}}.\mathrm{\text{V\scriptsize ERIFIER}\text{S\scriptsize LOW}}}
\newcommand*{\NARKVerifierFast}{\mathrm{\text{NARK}}.\mathrm{\text{V\scriptsize ERIFIER}\text{F\scriptsize AST}}}
\newcommand*{\IVCProver}{\mathrm{\text{IVC}}.\Prover}
\newcommand*{\IVCVerifier}{\mathrm{\text{IVC}}.\Verifier}
\newcommand*{\AS}{\text{AS}}
\newcommand*{\ASGenerator}{\AS.\Generator}
\newcommand*{\ASSetup}{\AS.\Setup}
\newcommand*{\ASProver}{\AS.\Prover}
\newcommand*{\ASVerifier}{\AS.\Verifier}
\newcommand*{\ASDecider}{\AS.\Decider}
\newcommand*{\PC}{\text{PC}}
\newcommand*{\PCSetup}{\PC.\Setup}
\newcommand*{\PCTrim}{\PC.\Trim}
\newcommand*{\PCCommit}{\PC.\Commit}
\newcommand*{\PCOpen}{\PC.\mathrm{\text{O\scriptsize PEN}}}
\newcommand*{\PCCheck}{\PC.\mathrm{\text{C\scriptsize HECK}}}
\newcommand*{\PCDL}{\text{PC}_{\text{DL}}}
\newcommand*{\PCDLSetup}{\PCDL.\Setup}
\newcommand*{\PCDLTrim}{\PCDL.\Trim}
\newcommand*{\PCDLCommit}{\PCDL.\Commit}
\newcommand*{\PCDLOpen}{\PCDL.\mathrm{\text{O\scriptsize PEN}}}
\newcommand*{\PCDLSuccinctCheck}{\PCDL.\mathrm{\text{S{\scriptsize UCCINCT}C{\scriptsize HECK}}}}
\newcommand*{\PCDLCheck}{\PCDL.\mathrm{\text{C\scriptsize HECK}}}
\newcommand*{\ASDL}{\text{AS}_{\text{DL}}}
\newcommand*{\ASDLSetup}{\ASDL.\Setup}
\newcommand*{\ASDLIndexer}{\ASDL.\mathrm{\text{I\scriptsize NDEXER}}}
\newcommand*{\ASDLCommonSubroutine}{\ASDL.\mathrm{\text{C{\scriptsize OMMON}S{\scriptsize UBROUTINE}}}}
\newcommand*{\ASDLProver}{\ASDL.\mathrm{\text{P\scriptsize ROVER}}}
\newcommand*{\ASDLVerifier}{\ASDL.\mathrm{\text{V\scriptsize ERIFIER}}}
\newcommand*{\ASDLDecider}{\ASDL.\mathrm{\text{D\scriptsize ECIDER}}}
\newcommand*{\CM}{\mathrm{\text{CM}}}
\newcommand*{\CMSetup}{\CM.\Setup}
\newcommand*{\CMTrim}{\CM.\Trim}
\newcommand*{\CMCommit}{\CM.\Commit}

\newcommand*\Result{\mathbf{Result}}
\newcommand*\Option{\mathbf{Option}}
\newcommand*\Acc{\mathbf{Acc}}
\newcommand*\Instance{\mathbf{Instance}}
\newcommand*\Proof{\mathbf{Proof}}
\newcommand*\Witness{\mathbf{Witness}}
\newcommand*\PublicInfo{\mathbf{PublicInputs}}
\newcommand*\PublicInputs{\mathbf{PublicInputs}}
\newcommand*\Circuit{\mathbf{Circuit}}
\newcommand*\AccHiding{\mathbf{AccHiding}}
\newcommand*\EvalProof{\mathbf{EvalProof}}

\newcommand{\plonkm}{\mathcal{P}\textfrak{lon}\mathcal{K}}
\newcommand{\plonk}{$\plonkm$}
\newcommand{\plookup}{$\textfrak{plookup}$}
\newcommand{\pcell}{\textfrak{p}}
\newcommand{\plonkup}{\plonk$\textfrak{up}$}
\newcommand{\Surkal}{\text{Surkål}}
\newcommand{\SurkalArithmetize}{\Surkal.\mathrm{\text{A\scriptsize RITHMETIZE}}}
\newcommand{\SurkalProver}{\Surkal.\mathrm{\text{P\scriptsize ROVER}}}
\newcommand{\SurkalVerifier}{\Surkal.\mathrm{\text{V\scriptsize ERIFIER}}}

\newcommand{\Unit}{\mathbf{1}}
\newcommand{\maybe}[2]{ \left\{ #1 \middle\vert #2 \right.}
\newcommand{\abst}[1]{ \hat{#1} }
\newcommand{\avec}[1]{\abst{\vec{#1}}}
\newcommand{\set}[1]{ \left\{ #1 \right\}}
\newcommand{\build}[3]{\left\llbracket #1 \right\rrbracket^{#2}_{#3}}
\newcommand{\AbsCirc}{\text{ACirc}}
\newcommand{\Ggt}{\text{Gadget}}
\makeatletter
\newcommand{\ggt}[2]{%
  \text{#1}(
    \def\ggt@first{1}%
    \@for\next:=#2\do{%
      \ifnum\ggt@first=1
        \abst{\next}%
        \def\ggt@first{0}%
      \else
        ,\abst{\next}%
      \fi
    }%
  )
}
\newcommand{\ggtw}[2]{%
  \text{#1}(
    \def\ggt@first{1}%
    \@for\next:=#2\do{%
      \ifnum\ggt@first=1
        \next%
        \def\ggt@first{0}%
      \else
        ,\next%
      \fi
    }%
  )
}
\newcommand{\ggtu}[2]{%
  #1(
    \def\ggt@first{1}%
    \@for\next:=#2\do{%
      \ifnum\ggt@first=1
        \next%
        \def\ggt@first{0}%
      \else
        ,\next%
      \fi
    }%
  )
}
\makeatother
\newcommand{\inr}{\text{inr}}
\newcommand{\inl}{\text{inl}}
\newcommand{\gpair}[2]{ \underline{#1\ #2} }
\newcommand{\Ops}{\text{Op}}
\newcommand{\AState}{\text{AState}}
\newcommand{\RState}{\text{RState}}
\newcommand{\VMap}{\text{VMap}}
\newcommand{\pto}{\rightharpoonup}
\newcommand{\Wire}{\text{Wire}}
\newcommand{\WireType}{\text{WTy}}
\newcommand{\wire}[2]{\underline{#1 : #2}}
\newcommand{\new}{\text{new}}
\newcommand{\ty}{\text{ty}}
\newcommand{\id}{\text{id}}
\newcommand{\idx}{\text{idx}}
\newcommand{\gin}{\text{in}}
\newcommand{\tin}[1]{\vec{t}^{\vec{in}\ifstrempty{#1}{}{,#1}}}
\newcommand{\tout}[1]{\vec{t}^{\vec{out}\ifstrempty{#1}{}{,#1}}}
\newcommand{\out}{\text{out}}
\newcommand{\entries}{\text{pairs}}
\newcommand{\aput}{\text{put}}
\newcommand{\aget}{\text{get}}
\newcommand{\otherwise}{\text{otherwise}}
\newcommand{\lift}{\text{lift}}
\DeclareMathOperator*{\opcirc}{\circ}
\DeclareMathOperator*{\opcat}{\cat}
\newcommand{\Slot}{\text{Slot}}
\newcommand{\Selector}{\text{Selector}}
\newcommand{\SlotNSelector}{\text{Slot} + \text{Selector}}
\newcommand{\Column}{\text{Column}}
\newcommand{\eval}{\text{eval}}
\newcommand{\refg}{\text{ref}}
\newcommand{\Base}{\text{Base}}
\newcommand{\ctrn}{\text{ctrn}}
\newcommand{\term}{\text{term}}
\newcommand{\IndexMap}{\text{IMap}}
\newcommand{\Uni}{\mathcal{U}}
\newcommand{\TraceTable}{\text{TraceTable}}
\newcommand{\Cell}{\text{Cell}}
\newcommand{\PreTable}{\text{PreTable}}
\newcommand{\OpGroup}{\text{OGrp}}
\newcommand{\OpCollection}{\text{OCol}}
\newcommand{\Grp}{\text{Grp}}
\newcommand{\Col}{\text{Col}}
\newcommand{\Spec}{\text{Spec}}
\newcommand{\FMap}{\text{FMap}}
\newcommand{\Eqn}{\text{Eqn}}
\newcommand{\AWire}{\text{AWire}}
\newcommand{\lin}{\text{lin}}
\newcommand{\Arg}{\text{Arg}}
\newcommand{\base}{\text{base}}
\newcommand{\Input}{\text{Input}}
\newcommand{\pset}[1]{\mathcal{P}(#1)}
\newcommand{\Rel}{\text{Rel}}
